\documentclass[usegeometry, paper=a4, parskip=half, numbers=enddot,  fontsize=11pt, toc=sectionentrywithdots]{scrartcl}

%!Deutsches Sprachpaket
\usepackage[ngerman]{babel}

\usepackage{pgf}
\usepackage{float}
\usepackage{tikz}
%\usepackage{caption} %nicht notwendig für captions
%\usepackage{lipsum}
\usepackage{forest}
%!\usetikzlibrary;
\begin{document}

\begin{flushleft}
Freie Universität Berlin\\
Institut für Deutsche und Niederländische Philologie\\

% Zeit\\
% Raum\\
Wintersemester 2022/23
\end{flushleft}

\vspace{0.2cm}

\begin{center}
\LARGE{Phrasenschemata}\\[10pt]
\Large{Themenblatt}\\[10pt]
\large{XXXX name}\\[3pt]
\small{\pgfrdfhref{mailto:x@xtdialup.fu-berlin.de}{noadressfu-berlin.de}}\\[10pt]
\large{\today}\\[30pt]
\end{center}


\section{essai}
1st line\\[14pt]

\begin{figure}[h]\label{figure1}
\centering

\begin{tikzpicture}
\node {root} 
child {node {left}} 
child {node {right} 
child {node {child}} 
child {node {child}}};
%!\caption{this is a figure}
%!\label{figure1}
  \end{tikzpicture}
  \caption{A picture}
  \end{figure}

blub undsoweiter
for reference see: %\cref{figure1}
\begin{figure}
\begin{forest}
[NP, dashed
	[Art|Pro, dashed]
	[AP, dashed, double]
	[N]
]
\end{forest}
\end{figure}

\begin{figure}
\begin{forest}
[PP, circle, draw
	[AdvP|NP, circle, draw, dashed
		[links, no edge, tier=3]
	]
	[TEXT1, circle, draw, tier=1
		[auf, no edge, tier=2
			[neben, no edge, tier=3,
				[ins, no edge, tier=4]
			]
		]
	]
	[NP, circle, draw, tier=1
		[der Einfahrt, no edge, tier=2
			[der Einfahrt, no edge, tier=3
				[Gebäude, no edge, tier=4]
			]
		]
	]
]
\end{forest}
\end{figure}

\end{document}
