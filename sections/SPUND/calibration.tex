%\usepackage[linguistics]{forest}

%\usetikzlibrary

%!---- aleks input
%################### LINGUISTIK: INFO- UND ÜBUNGSBLATT ####################

%%%%%%%%%% GRUNDLAGEN %%%%%%%%%%%%%%%%%%%%%%%%%%%%%%%%%%%%%%%%%%%%%%%%%%%%%

\documentclass[usegeometry, paper=a4, parskip=half, numbers=enddot,  fontsize=11pt, toc=sectionentrywithdots]{scrartcl}
\usepackage{pgf}
%\usepackage{float}
\usepackage{pict2e}
%\usepackage{amsthm}
\usepackage{caption}
\usepackage{lipsum}

%!Deutsches Sprachpaket
\usepackage[ngerman]{babel}

%!Deutsche Anführungszeichen
\usepackage[german=guillemets]{csquotes}

%!Zeilenabstand
\usepackage[singlespacing]{setspace}

%!Textauszeichnung (hier fürs Durchstreichen)
\usepackage{soul}

%%%%%%%%%% TEXTGESTALTUNG %%%%%%%%%%%%%%%%%%%%%%%%%%%%%%%%%%%%%%%%%%%%%%%%%

%!Große Textpassagen auskommentieren
\usepackage{comment}

%!Grafiken & Bilder
\usepackage{graphics}

%!Textfärbung
\usepackage{xcolor}

\definecolor{airforceblue}{rgb}{0.36, 0.54, 0.66}
\definecolor{babyblue}{rgb}{0.54, 0.81, 0.94}
\definecolor{bittersweet}{rgb}{1.0, 0.44, 0.37}
\definecolor{dartmouthgreen}{rgb}{0.05, 0.5, 0.06}

%%%%%%%%%% TABELLEN %%%%%%%%%%%%%%%%%%%%%%%%%%%%%%%%%%%%%%%%%%%%%%%%%%%%%%%

%!Tabularx-Tabellen
\usepackage{tabularx}

%!Tabellenlinien
\usepackage{booktabs}

%!Ausrichtung des Tabellen- bzw. Figurentitels
\usepackage[format=plain,singlelinecheck=false]{caption}

%!Abstand zwischen Tabelle/Abbildung und Beschriftung
%!Syntax: '\setlength{\belowcaptionskip}{<X>pt}'
%!Für 'X' natürliche Zahlen.
\setlength{\belowcaptionskip}{5pt}

%!Zusätzliche Positionsoption `[H]' zum Stoppen des Gleitens
\usepackage{float}

%!Zusätzliche Formatierung für Tabellen
\usepackage{array}

%!Befehl für Änderung der Abstände zwischen Zeilen in Tabellen:
%{\renewcommand{\arraystretch}{3.0} <table-Umgebung>}

%!Befehl für Abstand zwischen Spalten in Tabellen:
%{\setlength{\tabcolsep}{12pt} <table-Umgebung>}

%%%%%%%%%% LOGIK & MATHE %%%%%%%%%%%%%%%%%%%%%%%%%%%%%%%%%%%%%%%%%%%%%%%%%%

%---- Allgemein -----%

%!Befehle und Umgebungen für Mathe: Enthät u.a. Fix von amsmath, und graphicx (zum Einfügen von Bildern und Grafiken)
\usepackage{mathtools}

%!Weitere Mathezeichen. Enthält amsfonts
\usepackage{amssymb}

%!Theoreme
\usepackage{amsthm}

%%%%%%%%%% LINGUISTIK %%%%%%%%%%%%%%%%%%%%%%%%%%%%%%%%%%%%%%%%%%%%%%%%%%%%%

%---- Allgemein ----%

%!Linguistik-Grundlagen-Paket: Bequeme Nummerierung, Formatierung etc.
%!ACHTUNG: Muss unbedingt nach Font-Paketen wie bspw. tipa gelanden werden
\usepackage{linguex}

%!Den Bindestrich zwischen Subelementen innerhalb einer linguex-Umgebung entfernen
\renewcommand{\firstrefdash}{}

%!Baumdiagramme
\usepackage[linguistics]{forest}

%!forest-Optionen in Baumdiagrammen
%!'[,phantom]': Nebeneinanderstellen von Strukturen mittels Erstellung eines Scheinmutterknotens
%!'[<>,for children={fit=band}]': Leerer Platz, wenn kein Element/Kind
%!'[<>,tier=mother]': Änderung der Anordnung (vertikal) innerhalb der Struktur
%!'[<>,name=phrase]': Benennung für das Zeichnen von Linien
%!'\draw (<name1>) -- (<name2>);': Linienzeichnung
%!'[{},point]: für das Abwinkeln von Ästen. Hierfür muss 'point/.style={coordinate,},' zu Beginn der forest-Umgebung stehen, noch vor der Struktur.

%---- Definitionen ----%

%-Theoreme-%

\theoremstyle{definition}
\newtheorem{D}{Definition}
\newtheorem{B}{Beispiel}
\newtheorem{Satz}{Satz}
\newtheorem{N}{Notation}

%!Theoremstil definition mit Zeilenumbruch
\newtheoremstyle{break}
{\topsep}%
{\topsep}%
{\normalfont}%
{}%
{\bfseries}%
{.}%
{\newline}
{}%

\theoremstyle{break}
\newtheorem{R}{Regularität}
\newtheorem{Bsp}[B]{Beispiel}
\newtheorem{Sch}{Schema}

%-Baumdiagramm-Notation-%

%!forest-Baumdiagramm: Standard
\newcommand{\lingforest}[2]{%
	\noindent\makebox[\textwidth]{
		\begin{forest}
			where n children=0{tier=word}{},
			for tree={l=#1pt}
			#2
	\end{forest}}
}

%!forest-Baumdiagramm: Pittner-Notation
\newcommand{\lingforestpittner}[3]{%
	\noindent\makebox[\textwidth]{
		\begin{forest}
			where n children=0{tier=word}{},
			for tree={l=#1pt}
			[,phantom
			#2
			#3
			]
	\end{forest}}
}

%-Klammer-Notation (Pittner)-%

%!Klammern
\newcommand{\klammer}[2]{[#1]\textsubscript{\textbf{#2}}}
\newcommand{\Klammer}[2]{$\biggl[$#1$\biggr]$\textsubscript{\textbf{#2}}}
\newcommand{\KKlammer}[2]{$\Biggl[$#1$\biggr]$\textsubscript{\textbf{#2}}}
\newcommand{\pos}[1]{\textsubscript{\textbf{#1}}}

%!Grüne Klammern
\newcommand{\greenklammer}[2]{\textcolor{dartmouthgreen}{[}#1\textcolor{dartmouthgreen}{]}\textsubscript{\textbf{#2}}}
\newcommand{\greenKlammer}[2]{$\color{dartmouthgreen}\biggl[$#1$\color{dartmouthgreen}\biggr]$\textsubscript{\textbf{#2}}}
\newcommand{\greenKKlammer}[2]{$\color{dartmouthgreen}\Biggl[$#1$\color{dartmouthgreen}\Biggr]$\textsubscript{\textbf{#2}}}
%\newcommand{\greenpos}[1]{\textsubscript{\textbf{\textcolor{dartmouthgreen}{#1}}}}
\newcommand{\green}[1]{\textcolor{dartmouthgreen}{#1}}

%!Klammern: Übungen
\newcommand{\exklammer}[1]{[#1]\textsubscript{\subleer}}
\newcommand{\exKlammer}[1]{$\biggl[$#1$\biggr]$\textsubscript{\subleer}}
\newcommand{\exKKlammer}[1]{$\Biggl[$#1$\biggr]$\textsubscript{\subleer}}

%!Leerstelle im Subskript
\newcommand{\subleer}{\textsubscript{\textlarger[4]{\_}}}

%-Skripte-%

%!Oberskript
\newcommand{\oberskript}[2]{[$\overset{\textbf{#2}}{\text{#1}}$]}

%!Unterskript
\newcommand{\unterskript}[2]{$\underset{\textbf{#2}}{\text{\ul{#1}}}$}

%%%%%%%%%% ENDE %%%%%%%%%%%%%%%%%%%%%%%%%%%%%%%%%%%%%%%%%%%%%%%%%%%%%%%%%%%

%!Seitenränder
%\usepackage[top=2cm, bottom=4cm]{geometry}

%!PDF-Einbindung und Formatierung. Auch zum Setzen interner Verlinkungen
%!Zur Konfliktvermeidung zuletzt einbinden (Ausnahme: cleveref)
\usepackage[pdfauthor={Aleksandr Schamberger}, pdftitle={Themenblatt: Phrasenschemata}]{hyperref}

%!Bequemes/effizientes Referenzieren
%!Als aller, aller letztes einbinden
\usepackage[noabbrev, german]{cleveref}

%!Linguistik-Theoreme
\crefname{D}{Definition}{Definitionen}
\crefname{B}{Beispiel}{Beispiele}
\crefname{Satz}{Satz}{Sätze}
\crefname{N}{Notation}{Notationen}
\crefname{R}{Regularität}{Regularität}
\crefname{Bsp}{Beispiel}{Beispiele}
\crefname{Sch}{Schema}{Schemata}

%!---- aleks header end

